\documentclass[12pt]{article}
\usepackage{amsmath}
\usepackage{amssymb}
\usepackage{geometry}
\geometry{a4paper, margin=1in}

\title{Homework 2 - Electromagnetism}
\author{Son Nguyen \\
I pledge my honor that I have abided by the Stevens Honor System.}
\date{\today}

\begin{document}

\maketitle
\section*{Problem 1}
\[\vec{v} = x^2 \hat{x} + 2yz \hat{y} +y^2 \hat{z}\]
\begin{itemize}
	\item \((0,0,0) \rightarrow (1,0,0) \rightarrow (1,1,0) \rightarrow (1,1,1)\)
	      \begin{align*}
		      (0,0,0) \rightarrow (1,0,0) & = \int_{(0,0,0)}^{(1,0,0)} \vec{v} \cdot dl = \int_{0}^{1} x^2 dx = \frac{1}{3}                   \\
		      (1,0,0) \rightarrow (1,1,0) & = \int_{(1,0,0)}^{(1,1,0)} \vec{v} \cdot dl = \int_{0}^{1} 2yz dy = 0 \quad (z = 0)               \\
		      (1,1,0) \rightarrow (1,1,1) & = \int_{(1,1,0)}^{(1,1,1)} \vec{v} \cdot dl = \int_{0}^{1} y^2 dz = \int_{0}^{1}dz = 1 \quad(y=1) \\
		      \text{Total}                & = \frac{1}{3} + 0 + 1 = \frac{4}{3}
	      \end{align*}
	\item \((0,0,0) \rightarrow (0,0,1) \rightarrow (0,1,1) \rightarrow (1,1,1)\)
	      \begin{align*}
		      (0,0,0) \rightarrow (0,0,1) & = \int_{(0,0,0)}^{(0,0,1)} \vec{v} \cdot dl = \int_{0}^{1} y^2 dz = 0 \quad (y = 0)                     \\
		      (0,0,1) \rightarrow (0,1,1) & = \int_{(0,0,1)}^{(0,1,1)} \vec{v} \cdot dl = \int_{0}^{1} 2yz dy = \int_{0}^{1} 2ydy = 1 \quad (z = 1) \\
		      (0,1,1) \rightarrow (1,1,1) & = \int_{(0,1,1)}^{(1,1,1)} \vec{v} \cdot dl = \int_{0}^{1} x^2 dx = \frac{1}{3}                         \\
		      \text{Total}                & = 0 + 1 + \frac{1}{3} = \frac{4}{3}
	      \end{align*}
	\item straight line from \((0,0,0)\) to \((1,1,1)\)
	      Since it is a straight line, the function \(\vec{v}\) should be evaluated along a specific path parameterization \(l(t) = (t,t,t) \), this is because
	      all the variables are dependent on each other.
	      \[x = t, y = t, z = t, 0 \leq t \leq 1\]
	      \[dx = dt, dy = dt, dz = dt\]
	      \begin{align*}
		      \int_{(0,0,0)}^{(1,1,1)} \vec{v}(l(t)) \cdot dl & = \int_{0}^{1} t^2 \cdot 1 dt + \int_{0}^{1} 2t^2 \cdot 1 dt \int_{0}^{1} t^2 \cdot 1 dt \\
		                                                      & = \int_{0}^{1} t^2 + 2t^2 + t^2 dt = \int_{0}^{1} 4t^2 dt = \frac{4}{3}
	      \end{align*}
\end{itemize}
\section*{Problem 2}
\[T= z^2\]
\[\int_{v}^{} T dt\]
At first, in the \((x,y)\) coordinate, we have the line go from \((1,0,0) \rightarrow (0,1,0)\). The line can be written as \(y = 1 - x\). Therefore the bounds of the integral are:
\[0 \leq x \leq 1, 0 \leq y \leq x - 1 \quad \text{(as \(x\) goes from \(0 \rightarrow 1\), \(y\) goes from \(0 \rightarrow (1-x)\))}\]
For the plane going bounded by 3 poinnts \((1,0,0), (0,1,0), (0,0,1)\), the equation of the plane is \(x+y+z = 1\) or \
\(z = 1 - x - y\). The bound of this integral is:
\[0 \leq z \leq 1 - x - y \quad \text{(z goes from \(0 \rightarrow 1 - x - y\))}\]
Therefore, the integral is:
\begin{align*}
	\int_{0}^{1} \int_{0}^{1-x} \int_{0}^{1-x-y} T dt & = \int_{0}^{1} \int_{0}^{1-x} \int_{0}^{1-x-y} z^2 dz dy dx                \\
	                                                  & = \int_{0}^{1} \int_{0}^{1-x} \left[\frac{z^3}{3}\right]_{0}^{1-x-y} dy dx \\
	                                                  & = \int_{0}^{1} \int_{0}^{1-x} \frac{(1-x-y)^3}{3} dy dx                    \\
	                                                  & = \frac{1}{3} \int_{0}^{1} \int_{0}^{1-x} (1-x-y)^3 dy dx                  \\
\end{align*}
Substitute \( u = 1-x-y\) and \(du = -dy\). The new lower bound is \(u = 1-x\) and the upper bound is \(u = 1-x-(1-x) = 0\).
\begin{align*}
	\frac{-1}{3} \int_{0}^{1} \int_{1-x}^{0} u^3 du dx &= \frac{-1}{3} \int_{0}^{1} \left[\frac{u^4}{4}\right]_{1-x}^{0} dx \\
	&= \frac{-1}{3} \int_{0}^{1} \frac{-(x-1)^4}{4} dx \\
	&= \frac{1}{12} \int_{0}^{1} (x-1)^4 dx \\
	&= \frac{1}{12} \left[\frac{(x-1)^5}{5}\right]_{0}^{1} \\
	&= \frac{1}{60}
\end{align*}
\section*{Problem 3}
\[\vec{v} = (xy) \hat{x} + (2yz)\hat{y} + (3xz) \hat{z}\]
Divergence theorem states that:
\[\int_{v}^{}(\nabla \cdot v)d\tau = \int_{s}^{} v\cdot da \]
\begin{align*}
	\int_{v}^{}\nabla \cdot \vec{v} d\tau & = \int_{0}^{2} \int_{0}^{2} \int_{0}^{2} \left[\frac{\partial(xy)}{\partial x} + \frac{\partial(2yz)}{\partial y} + \frac{\partial (3xz)}{\partial z}\right] dx dy dz \\
	&= \int_{0}^{2} \int_{0}^{2} \int_{0}^{2} (y + 2z + 3x) dx dy dz \\
	&= \int_{0}^{2} \int_{0}^{2} \left[yx + 2zx + \frac{3x^2}{2}\right]_{0}^{2} dy dz \\
	&= \int_{0}^{2} \int_{0}^{2} (2y + 4z + 6) dy dz \\
	&= \int_{0}^{2} \left[y^2 + 4yz + 6y\right]_{0}^{2} dz \\
	&= \int_{0}^{2} (4 + 8z + 12) dz \\
	&= \left[4z + 4z^2 + 12z\right]_{0}^{2} \\
	&= 8 + 16 + 24 = 48
\end{align*}
For the surface integral, we have 6 faces. 2 top and bottom faces on the (xy plane with z = 0 and z = 2), 2 faces on the xz plane (y = 0 and y = 2), and 2 faces on the yz plane (x = 0 and x = 2). 
\begin{align*}
	\int_{s}^{} v\cdot da &= \int_{0}^{2} \int_{0}^{2} v_{z=2} dx dy - \int_{0}^{2} \int_{0}^{2} v_{z=0} dx dy  + \int_{0}^{2} \int_{0}^{2} v_{x=2} dz dy - \int_{0}^{2} \int_{0}^{2} v_{x=0} dz dy \\
	& +\int_{0}^{2} \int_{0}^{2} v_{y=2} dz dx - \int_{0}^{2} \int_{0}^{2} v_{y=0} dz dx \\
	&= \int_{0}^{2} \int_{0}^{2} 6x dx dy - \int_{0}^{2} \int_{0}^{2} 0 dx dy  + \int_{0}^{2} \int_{0}^{2} 2y dy dz - \int_{0}^{2} \int_{0}^{2} 0 dz dy \\
	& +\int_{0}^{2} \int_{0}^{2} 4z dx dz - \int_{0}^{2} \int_{0}^{2}0 dx dz \\
	&= \int_{0}^{2} \int_{0}^{2} 6x dx dy + \int_{0}^{2} \int_{0}^{2} 2y dy dz + \int_{0}^{2} \int_{0}^{2} 4z dx dz \\
	&= \int_{0}^{2} \left[3x^2\right]_{0}^{2} dy + \int_{0}^{2} \left[y^2\right]_{0}^{2} dz + \int_{0}^{2} \left[2z^2\right]_{0}^{2} dx \\
	&= \int_{0}^{2} 12 dy + \int_{0}^{2} 4 dz + \int_{0}^{2} 8 dx \\
	&= \left[12y\right]_{0}^{2} + \left[4z\right]_{0}^{2} + \left[8x\right]_{0}^{2} \\
	&= 24 + 8 + 16 = 48
\end{align*}
Therefore, the two integrals are equal.
\section*{Problem 4}
\[\int_{S}^{} (\nabla \times v) \cdot da = \oint_P v \cdot dl\]
Since we only include the (yz) plane.
\begin{align*}
	\int_{S}^{} (\nabla \times v) \cdot dydz &= \int_{0}^{2} \int_{0}^{2-z} \begin{vmatrix}
		\hat{x} & \hat{y} & \hat{z} \\
		\frac{\partial}{\partial x} & \frac{\partial}{\partial y} & \frac{\partial}{\partial z} \\
		xy & 2yz & 3xz
	\end{vmatrix} dy dz \\
	&= \int_{0}^{2} \int_{0}^{2-z} \begin{vmatrix}
		\hat{x} & 0 & 0 \\
		\frac{\partial}{\partial x} & \frac{\partial}{\partial y} & \frac{\partial}{\partial z} \\
		xy & 2yz & 3xz
	\end{vmatrix} dy dz \quad (\text{since we only include the (yz) plane}) \\
	&= \int_{0}^{2} \int_{0}^{2-z} \left[\frac{\partial(3xz)}{\partial y} - \frac{\partial(2yz)}{\partial z}\right] dy dz \\
	&= \int_{0}^{2} \int_{0}^{2-z} 0 - 2y dy dz \\
	&= \int_{0}^{2} \left[-y^2\right]_{0}^{2-z} dz \\
	&= \int_{0}^{2} -(2-z)^2 dz \\
	&= - \int_{0}^{2} 4 - 4z + z^2 dz \\
	&= - \left[4z - 2z^2 + \frac{z^3}{3}\right]_{0}^{2} \\
	&= -8 + 8 - \frac{8}{3} = -\frac{8}{3}
\end{align*}

Let line \(l1\) be the staright like from \((0,0,2) \rightarrow (0,0,0)\), line \(l2\) be the straight line from \((0,0,0) \rightarrow (0,2,0)\), 
and line \(l3\) be the straight line from \((0,2,0) \rightarrow (0,0,2)\). We can parameterized the lines as:
\begin{align*}
	l1(t) &= (0,0,2-t) \quad 0 \leq t \leq 2 \\
	l2(t) &= (0,t,0) \quad 0 \leq t \leq 2 \\
	l3(t) &= (0,2-t,t) \quad 0 \leq t \leq 2
\end{align*}

\begin{align*}
	\oint_P v \cdot dl &= \int_{0}^{2} v(l1(t)) \cdot dl1(t) + \int_{0}^{2} v(l2(t)) \cdot dl2(t) + \int_{0}^{2} v(l3(t)) \cdot dl3(t) \\
	&= \int_{0}^{2} v(0,0,2-t) \cdot (0,0,-1) dt + \int_{0}^{2} v(0,t,0) \cdot (0,1,0) dt + \int_{0}^{2} v(0,2-t,t) \cdot (0,-1,1) dt \\
	&= \int_{0}^{2} -3(2-t)\cdot 0 dt \int_{0}^{2} 2(t)(0) dt + \int_{0}^{2} 2(t)(-1)(2-t) + 3(t)(0)dt \\
	&= 0 + 0 + \int_{0}^{2} -2t(2-t) dt \\
	&= \int_{0}^{2} -4t + 2t^2 dt \\
	&= \left[-2t^2 + \frac{2t^3}{3}\right]_{0}^{2} \\
	&= -8 + \frac{16}{3} = -\frac{8}{3}
\end{align*}
Therefore, the two integrals are equal.
\section*{Problem 5}
Prove:
\begin{itemize}
	\item \(\int_{S}^{} f(\nabla \times \vec{A}) \cdot d\vec{a} = \int_{S}^{} [\vec{A} \times (\nabla f)]\cdot d\vec{a} + \oint_P f\vec{A} \cdot d\vec{l}\)\\
	\\
	Left hand side: \\
	From equation (v) page 21, we have: \(\nabla \times (fA) = f(\nabla \times A) - A \times (\nabla f)\)\\
	\begin{align*}
		\int_{S}^{} f(\nabla \times \vec{A}) \cdot d\vec{a} &= \int_{S}^{} [\underbrace{\nabla \times (f\vec{A})}_{\text{Stoke's theorem}} + \vec{A} \times (\nabla f)] \cdot d\vec{a} \\
		&=  \int_{S}^{} [\vec{A} \times (\nabla f)]\cdot d\vec{a} + \oint_P f\vec{A} \cdot d\vec{l}
	\end{align*}
	
	\item \(\int_{V}^{}\vec{B} \cdot (\nabla \times \vec{A}) d\tau = \int_{V}^{}\vec{A}\cdot(\nabla \times \vec{B}) d\tau + \oint_S (\vec{A} \times \vec{B}) \cdot d\vec{a}\)
	\\ \\
	Left hand side: \\
	From equation (iv) page 21, we have: \(\nabla \cdot (A\times B) = B \cdot (\nabla \times A) - A \cdot (\nabla \times B)\)
	\begin{align*}
		\int_{V}^{}\vec{B} \cdot (\nabla \times \vec{A}) d\tau &= \int_{V}^{}[\underbrace{\nabla \cdot (\vec{A} \times \vec{B})}_{\text{Divergence Theorem}} + \vec{A} \cdot (\nabla \times \vec{B})] d\tau \\
		&= \int_{V}^{}\vec{A}\cdot(\nabla \times \vec{B}) d\tau + \oint_S(\vec{A} \times \vec{B}) \cdot d\vec{a}
	\end{align*}
\end{itemize}
\end{document}